%---------------------------------------------------------------------------------------------------
% Zusammenfassung
%---------------------------------------------------------------------------------------------------
% \newpage
%%\part{Schluss}
\chapter{Fazit und Ausblick}

  \begin{flushleft}
    Die Zielsetzung dieser Arbeit war es, den Einfluss hinsichtlich des Qualifikationsnivaus von Kita-Leitung auf die fr�hkindlichen Bildung in Kindertageseinrichtungen, zu beleuchten. Zu diesem Zweck wurde erst gekl�rt, welche Anforderungen, Aufgaben und T�tigkeiten Kita-Leitungen zu erf�llen haben. Im weiteren Verlauf wurde dann darauf eingegangen, welche Qualifikationsformen Leitungskr�fte in Deutschland innewohnen.
Dem folgte die  Auseinandersetzung mit nationalen und internationalen Forschungsergebnissen zu dem Thema Leitungskr�ften in Kindertageseinrichtungen.
  \end{flushleft}

  \begin{flushleft}
    Durch die Hausarbeit konnte gezeigt werden, welchen Stellenwert die Qualifikation auf die fr�hkindliche Bildung in Kindertageseinrichtungen hat. Gerade die internationalen Forschungsergebnisse, wie etwa die EPPE Studie, zeigen deutlich, dass ein h�her qualifizierteres Personal auch f�r mehr Qualit�t in der Einrichtung steht. Dies gilt im Besonderen, wenn die Kita-Leitung eine h�here Qualifizierung besitzt.\citep[vgl.][S.~4]{EPPE2004} Hierf�r scheint haupts�chlich das Verst�ndnis von Qualit�t und wie sie diese erreicht werden kann, verantwortlich zu sein.\citep[vgl.][S.~86]{Long2012}  Zu dem wurde aber auch deutlich gemacht, dass deutschlandweit  77\% der Kita-Leitungen keinen einschl�gigen Hochschulabschluss haben. Daher scheinen Fort- und Weiterbildungen derzeit eine gute L�sung zu sein um das Qualifikationsnivaus von Kita-Leitungen zu steigern.\citep[vgl.][S.~269]{Ausblick2014} Au"serdem scheint ein einschl�giger Hochschulabschluss nicht auszureichen, um dem Anforderungs-, Aufgaben- und T�tigkeitsprofil von Kita-Leitungen voll und ganz zu entsprechen. Daher sind auch hier Fort- und Weiterbildungen von gro"ser Bedeutung. \citep[vgl.][S.~269]{Ausblick2014}
  \end{flushleft}

  \newpage
  \begin{flushleft}
    Nun zu meinem pers�nlichen Fazit. Ich hatte mir von dieser Facharbeit erhofft, zwei Sachen zu erfahren. Zum einen war es mir wichtig zu erfahren, inwieweit das Thema in der Fach�ffentlichkeit thematisiert und diskutiert wird. Zum anderen wollte ich wissen, inwiefern zu meiner eingehenden Frage, wissenschaftliche Erkenntnisse existieren.
  \end{flushleft}

  \begin{flushleft}
    Leider wurden nicht beide Ziele zu meiner vollsten Zufriedenheit erf�llt. Ich hatte zwar im Rahmen dieser Arbeit die Gelegenheit, viele Erkenntnisse �ber die Meinungen der Fach�ffentlichkeit zu erlangen, da es mir hierbei in keiner Weise an Fachliteratur gemangelt hat. Dadurch habe ich neue Erkenntnisse �ber die Wichtigkeit von Kita-Leitungen und ihren T�tigkeiten erlangt, die mir auch in meinem weiteren Berufsleben sehr hilfreich sein werden. Was mich aber �berrascht hat, war dass es noch auf nationaler Ebene wenig wissenschaftliche Ergebnisse zu diesem Thema gibt. Zwar sind die internationalen Studien sehr informativ, doch h�tte ich mir bei der zugeschriebenen \enquote{Schl�sselposition}, mehr nationale Studien erhofft.
  \end{flushleft}

  \begin{flushleft}
    Durch die Recherche f�r die vorliegende Arbeit haben sich Weiterf�hrende Fragen ergeben - Welchen Einfluss hat die Freistellung der Kita-Leitung auf die Qualit�t der fr�hkindlichen Bildung? Welche Rolle Spielt die Freistellung? und Welche Rolle spielt das Alter einer Kita-Leitung? Ich k�nnte mir gut vorstellen, diese Fragestellungen in meiner Bachelorthesis weiterzuverfolgen und diese mit der vorliegenden Arbeit zu verkn�pfen.
  \end{flushleft}
  \begin{flushleft}
  \end{flushleft}
