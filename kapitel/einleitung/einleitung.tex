%---------------------------------------------------------------------------------------------------
% Einf�hrung
%---------------------------------------------------------------------------------------------------
% \newpage
%%\part{Anfang}
\chapter{Einleitung}\label{sec:1}

  \begin{flushleft}
    Kita-Leitungen wurde lange keine besondere Rolle in der fr�hkindlichen Bildung zugeschrieben. Erst der seit einigen Jahren bestehende Diskurs �ber die Qualit�t und Wirksamkeit von Kindertagesbetreuung, hat die verschiedenen Akteure der fr�hkindlichen Bildung in den Blick der �ffentlichkeit ger�ckt. So stellen sich die Fragen wie qualitativ hochwertige Arbeit in der fr�hkindlichen Bildung aussehen soll und wie sie erreicht werden kann. Ein Ansatz sind Kataloge und Leitlinien die die Fachkr�fte in ihrer p�dagogischen Unterst�tzen sollen. Ebenfalls zur Debatte steht die H�herqualifizierung der Fachkr�fte, da dies zur Professionalisierung des Berufsbildes beitragen k�nnte. Ihrer heute zugeschriebenen Schl�sselposition verdankt die Kita-Leitung also den ver�nderten gesellschaftlichen Rahmbedingungen. \citep[vgl.][S.~14]{Lupe2014}
  \end{flushleft}

  \begin{flushleft}
    Das Berufsprofil der Kita-Leitung ist diffus wie auch komplex. Mitverantwortlichen f�r den diffusen Zustand sind die bundesweit unterschiedlichen Gesetze f�r Kitas und ihre F�hrungskr�fte. Auch die verschieden Tr�ger tun ihr �briges. Jeder Tr�ger hat andere Anforderungen, Aufgaben und T�tigkeiten die ihre Leitungskr�fte zu erf�llen haben. Nat�rlich kann grob gesagt werden welche Anforderungen, Aufgaben und T�tigkeiten Kita-Leitungen innewohnen. So muss sie sich selbst und ihre Mitarbeiterinnen f�hren und die Zusammenarbeit im Team gestalten. Au�erdem muss sie die Kita weiterentwickeln, das Umfeld beobachten, Trends erkennen und mit anderen Einrichtungen zusammenarbeiten. \citep[vgl.][S.~41]{LeadershipNon2008}
  \end{flushleft}

  \newpage

  \begin{flushleft}
    Im Verlauf dieser Arbeit werde ich folgender Frage nachgehen: Welchen Einfluss hat die Qualifikation der Kita-Leitung auf die fr�hkindlichen Bildung in Kindertageseinrichtungen? Am Anfang dieser Arbeit werde ich das Berufsprofil der Kita-Leitung unter die Lupe nehmen. Hierbei wird deutlich, wie Komplex dieser Beruf ist und das er viele Kompetenzen verlangt. Im Folgenden werde ich dann noch auf die verschiedenen Qualifikationsformen von Leitungskr�ften eingehen. Dies ist von besondere Bedeutung, da hier aufgezeigt wird welche Ausbildungsform am besten auf den Leitungsberuf vorbereitet. Au�erdem gebe ich einen kleinen Einblick in nationale wie auch internationale Forschungsergebnisse zu diesem Thema. Im letzten Kapitel fasse ich zusammen und gebe mein Fazit wieder.
  \end{flushleft}

  \begin{flushleft}
    Viele der Themen konnte ich nur anrei�en, da es sonst den Umfang dieser Arbeit �berschritten h�tte. Trotz allem hoffe ich, dass diese Hausarbeit dem Leser einen kleinen Einblick geben wird, welchen Einfluss die Qualifikation der Kita-Leitung auf die fr�hkindlichen Bildung in Kindertageseinrichtungen hat.
  \end{flushleft}

  \begin{flushleft}
    Des Weiteren m�chte ich darauf hinweisen, dass die Arbeit sich haupts�chlich auf Kita-Leitungen bezieht die freigestellt sind. Au�erdem verwende ich, der Einfachheithalber, immer die weibliche Form wenn z.B. von Mitarbeitern und Mitarbeiterinne oder �hnliches die Rede ist.
  \end{flushleft}

  \newpage
