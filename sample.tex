\documentclass[draft=false,paper=a4,twoside=false,fontsize=11pt,headsepline,BCOR10mm,DIV11,liststotoc]{scrbook}
% \usepackage{showframe}
\usepackage{titlesec}
% \def\chaptertitlename{Section}
% \titleformat{\chapter}{\normalfont\Large\bfseries}{\chaptertitlename\ \thechapter{:\ }}{0pt}{\Large}{}
\titleformat{\chapter}[hang]{\huge\bfseries}{\thechapter\quad}{0pt}{}
% \titlespacing{\chapter}{0pt}{-20pt}{30pt}
\titlespacing{\chapter}{0pt}{-3em}{2em}
\usepackage[ngerman,english]{babel}
\usepackage[latin1]{inputenc}
\usepackage[babel,german=quotes]{csquotes}
%% see http://www.tex.ac.uk/cgi-bin/texfaq2html?label=uselmfonts
\usepackage[T1]{fontenc}
%\usepackage[utf8]{inputenc}
\usepackage{eurosym}
\usepackage{libertine}
\usepackage{pifont}
\usepackage{microtype}
\usepackage{textcomp}
% \usepackage[intoc,german,prefix]{nomencl}
\usepackage{setspace}
\usepackage{makeidx}
\usepackage{listings}
\usepackage{natbib}
\usepackage[ngerman,colorlinks=true]{hyperref}
\usepackage{soul}
\usepackage{tabularx}
\usepackage{multirow}
\usepackage{multicol}
\usepackage{glossaries}
%%\usepackage{booktabs}
\usepackage{colortbl}
\usepackage[final]{pdfpages}
\usepackage[printer]{hawstyle}
% \usepackage{hawstyle}
\usepackage{caption}
\usepackage{lipsum} %% for sample text

\usepackage{float}
\restylefloat{table}
\restylefloat{figure}


\definecolor{middlegray}{rgb}{0.5,0.5,0.5}
\definecolor{lightgray}{rgb}{0.8,0.8,0.8}

%% define some colors
\colorlet{BackgroundColor}{gray!20}
\colorlet{KeywordColor}{blue}
\colorlet{CommentColor}{black!60}
%% for tables
\colorlet{HeadColor}{gray!60}
\colorlet{Color1}{blue!10}
\colorlet{Color2}{white}

%% configure colors
\HAWifprinter{
  \colorlet{BackgroundColor}{gray!20}
  \colorlet{KeywordColor}{black}
  \colorlet{CommentColor}{gray}
  % for tables
  \colorlet{HeadColor}{gray!60}
  \colorlet{Color1}{gray!40}
  \colorlet{Color2}{white}
}{}
\lstset{%
  language=JAVA,
  numbers=left,
  numberstyle=\tiny,
  stepnumber=1,
  tabsize=2,
  numbersep=-10pt,
  basicstyle=\ttfamily\small,
  keywordstyle=\color{KeywordColor}\bfseries,
  identifierstyle=\color{black},
  commentstyle=\color{middlegray},
  backgroundcolor=\color{BackgroundColor},
  captionpos=b,
  breaklines=true,
  fontadjust=true,
  flexiblecolumns=true
}
\lstset{escapeinside={(*@}{@*)}, % used to enter latex code inside listings
        morekeywords={uint32_t, int32_t}
}

% \usepackage{courier}
% \usepackage{caption}
% \lstset{
%   basicstyle=\footnotesize\ttfamily, % Standardschrift
%   numbers=left,               % Ort der Zeilennummern
%   numberstyle=\tiny,          % Stil der Zeilennummern
%   %stepnumber=2,               % Abstand zwischen den Zeilennummern
%   numbersep=5pt,              % Abstand der Nummern zum Text
%   tabsize=2,                  % Groesse von Tabs
%   extendedchars=true,         %
%   breaklines=true,            % Zeilen werden Umgebrochen
%   keywordstyle=\color{red},
%   %frame=b,
%   %        keywordstyle=[1]\textbf,    % Stil der Keywords
%   %        keywordstyle=[2]\textbf,    %
%   %        keywordstyle=[3]\textbf,    %
%   %        keywordstyle=[4]\textbf,   \sqrt{\sqrt{}} %
%   stringstyle=\color{white}\ttfamily, % Farbe der String
%   showspaces=false,           % Leerzeichen anzeigen ?
%   showtabs=false,             % Tabs anzeigen ?
%   xleftmargin=17pt,
%   framexleftmargin=17pt,
%   framexrightmargin=5pt,
%   framexbottommargin=4pt,
%   %backgroundcolor=\color{lightgray},
%   showstringspaces=false      % Leerzeichen in Strings anzeigen ?
% }
% %\DeclareCaptionFont{blue}{\color{blue}}
%
% %\captionsetup[lstlisting]{singlelinecheck=false, labelfont={blue}, textfont={blue}}
% \DeclareCaptionFont{white}{\color{white}}
% \DeclareCaptionFormat{listing}{\colorbox{HeadColor}{\parbox{\textwidth}{\hspace{15pt}#1#2#3}}}
% \captionsetup[lstlisting]{format=listing,labelfont=white,textfont=white, singlelinecheck=false, margin=0pt, font={bf,footnotesize}}



\ifpdfoutput{
  \hypersetup{bookmarksopen=false,bookmarksnumbered,linktocpage}
}{}

%% more fancy C++
\DeclareRobustCommand{\cxx}{C\raisebox{0.25ex}{{\scriptsize +\kern-0.25ex +}}}

\clubpenalty=10000
\widowpenalty=10000
\displaywidowpenalty=10000

% unknown hyphenations
\hyphenation{
}

%% recalculate text area
\typearea[current]{last}

\makeindex
%% \makenomenclature
\makeglossaries

\begin{document}
\selectlanguage{ngerman}

%%%%%
%% customize (see readme.pdf for supported values)
\HAWThesisProperties{Author={Shirin Bediako \\ MatrNr.: 2088775 \\ shirin.bediako@haw-hamburg.de}
                    ,AuthorTwo={Jelka Dr�cker \\ MatrNr.:2079988 \\ jelka.dr�cker@haw-hamburg.de}
                    ,Title={Das bikulturelle Identit"atsentwicklung afrodeutscher Kinder}
                    ,SubTitle={Bikulturalit"at als Einflussfaktor auf die Identit"atsentwicklung afrodeutscher Kinder}
                    ,EnglishTitle={Die bikulturelle Identit"atsentwicklung afrodeutscher Kinder}
                    ,ThesisType={Hausarbeit}
                    ,ExaminationType={Einf�hrung in die Familienberatung}
                    ,DegreeProgramme={Bildung und Erziehung in der Kindheit}
                    ,ThesisExperts={Prof.\ Dr.\ Wolfgang\ Hantel-Quitmann}
                    ,ReleaseDate={17. Oktober 2013}
                  }

%% title
\frontmatter

%% output title page
\maketitle

\onehalfspacing

%% add abstract pages
%% note: this is one command on multiple lines
% \HAWAbstractPage
%% German abstract
% {keywords...}%
% {text...}
%% English abstract
% {keywords...}%
% {text...}

\newpage
\singlespacing

\tableofcontents
\newpage
%% enable if these lists should be shown on their own page
% \listoftables
\listoffigures
% \lstlistoflistings

%% main
\mainmatter
\onehalfspacing
\newglossaryentry{Bewegungsangebote}{name={Bewegungsangebote},description={Unter Bewegungsangeboten werden, Bewegungsm�glichkeiten verstanden, die vom p�dagogischen Fachpersonal gestellten werden. Hierunter fallen r�umliche Gegebenheiten, sowie die Materialien die die Kinder nutzen k�nnen. Die Kinder haben hier unter p�dagogisch Aufsicht, die M�glichkeit frei zu spielen. Freispiel in vorbereiteter Umgebung.}}

\newglossaryentry{Bewegungsspiele}{name={Bewegungsspiele},description={Als Bewegungsspiele werden Spiel bezeichnet, die Bewegungst�tigkeiten von Kindern beschreibt, die sich aus Spielsituationen ergeben und meist selbstgesteuert sind.}}

\newglossaryentry{Egozentrismus}{name={Egozentrismus}, description={Ist die Tendenz, die Welt aus der eigenen Perspektive zu sehen.}}

\newglossaryentry{Empathie}{name={Empathie}, description={Als Empathie oder auch Einf�hlungsverm�gen wird die F�higkeit verstanden, Gef�hle bei anderen wahrzunehmen und sich in Gef�hlslage der andere Person hineinzuversetzen. Des Weiteren setzt es voraus, dass die eigenen Gef�hle von denen des anderen unterschieden werden. Empathie ist nicht zu verwechseln mit Gef�hlsansteckung oder Mitleid. Die F�higkeit empathisch zu denken, handeln und zu reagieren, k�nnen Kinder erst ausbilden, wenn sich das Selbstkonzept herausgebildet hat. Dieses erm�glicht die getrennte Wahrnehmung von dem \enquote{ich} und dem \enquote{anderen}.}}

\newglossaryentry{Freispiel}{name={Freispiel}, description={Das Freispiel ist ein wichtiger Bestandteil in der Tagesgestaltung im Kindergarten oder in der Kindertagesst�tte. Darunter wird verstanden,  Kindern die M�glichkeit zu bieten, w�hrend einer bestimmten Zeit, Spiele frei zu entwickeln und zu gestalten.}}

\newglossaryentry{Freundschaften}{name={Freundschaften}, description={Merkmal von Freundschaft ist die Freiwilligkeit, enge Gegegseitig positiv gesinnte Beziehung.}}

\newglossaryentry{Gleichaltrigen}{name={Gleichaltrigen}, description={siehe Peers}}

\newglossaryentry{Mitleid}{name={Mitleid}, description={Mitleid bezieht sich auf auf das Hineinversetzen in negative Gef�hle oder die Lebenslage einer anderen Person und dr�ckt sich in Anteilnahme, Kummer und Sorge aus.}}

\newglossaryentry{Peers}{name={Peers}, description={Peer sind Menschen von etwa gleichem Alter und Status.}}

\newglossaryentry{Prosoziales verhalten}{name={Prosoziales verhalten}, description={Das ist ein freiwilliges Verhalten, von denen andere einen positiven Effekt haben, wie zum Beispiel, jemanden zu tr�sten, etwas zu teilen oder zu helfen.}}

\newglossaryentry{soziometrischen Status}{name={soziometrischen Status}, description={Mit dem soziometrischer Status wird gemessen wie sehr ein Kind von seinen Peers als Gesamtgruppe gemocht wird.}}

\newglossaryentry{Selbstbild}{name={Selbstbild}, description={Das Selbstbild beschreibt das Bild, das sich ein Kind von seiner Person macht.}}

\newglossaryentry{Gefhlsansteckung}{name={Gef�hlsansteckung}, description={Der Begriff Gef�hlsansteckung, bezieht sich auf den emotionalen Zustand, in dem die Person durch die Identifikation sich in die gleiche emotionale Lage bring. Dieses Verhalten kann man schon bei S�uglingen beobachten. Wenn sie ein anderes Kind weinen h�ren, tun sie es ihm gleich.}}

%---------------------------------------------------------------------------------------------------
% Der erste Teil der Arbeit:
%---------------------------------------------------------------------------------------------------
\typeout{===== File: EINLEITUNG}
%---------------------------------------------------------------------------------------------------
% Einf�hrung
%---------------------------------------------------------------------------------------------------
% \newpage
%%\part{Anfang}
\chapter{Einleitung}\label{sec:1}

  \begin{flushleft}
    Kita-Leitungen wurde lange keine besondere Rolle in der fr�hkindlichen Bildung zugeschrieben. Erst der seit einigen Jahren bestehende Diskurs �ber die Qualit�t und Wirksamkeit von Kindertagesbetreuung, hat die verschiedenen Akteure der fr�hkindlichen Bildung in den Blick der �ffentlichkeit ger�ckt. So stellen sich die Fragen wie qualitativ hochwertige Arbeit in der fr�hkindlichen Bildung aussehen soll und wie sie erreicht werden kann. Ein Ansatz sind Kataloge und Leitlinien die die Fachkr�fte in ihrer p�dagogischen Unterst�tzen sollen. Ebenfalls zur Debatte steht die H�herqualifizierung der Fachkr�fte, da dies zur Professionalisierung des Berufsbildes beitragen k�nnte. Ihrer heute zugeschriebenen Schl�sselposition verdankt die Kita-Leitung also den ver�nderten gesellschaftlichen Rahmbedingungen. \citep[vgl.][S.~14]{Lupe2014}
  \end{flushleft}

  \begin{flushleft}
    Das Berufsprofil der Kita-Leitung ist diffus wie auch komplex. Mitverantwortlichen f�r den diffusen Zustand sind die bundesweit unterschiedlichen Gesetze f�r Kitas und ihre F�hrungskr�fte. Auch die verschieden Tr�ger tun ihr �briges. Jeder Tr�ger hat andere Anforderungen, Aufgaben und T�tigkeiten die ihre Leitungskr�fte zu erf�llen haben. Nat�rlich kann grob gesagt werden welche Anforderungen, Aufgaben und T�tigkeiten Kita-Leitungen innewohnen. So muss sie sich selbst und ihre Mitarbeiterinnen f�hren und die Zusammenarbeit im Team gestalten. Au�erdem muss sie die Kita weiterentwickeln, das Umfeld beobachten, Trends erkennen und mit anderen Einrichtungen zusammenarbeiten. \citep[vgl.][S.~41]{LeadershipNon2008}
  \end{flushleft}

  \newpage

  \begin{flushleft}
    Im Verlauf dieser Arbeit werde ich folgender Frage nachgehen: Welchen Einfluss hat die Qualifikation der Kita-Leitung auf die fr�hkindlichen Bildung in Kindertageseinrichtungen? Am Anfang dieser Arbeit werde ich das Berufsprofil der Kita-Leitung unter die Lupe nehmen. Hierbei wird deutlich, wie Komplex dieser Beruf ist und das er viele Kompetenzen verlangt. Im Folgenden werde ich dann noch auf die verschiedenen Qualifikationsformen von Leitungskr�ften eingehen. Dies ist von besondere Bedeutung, da hier aufgezeigt wird welche Ausbildungsform am besten auf den Leitungsberuf vorbereitet. Au�erdem gebe ich einen kleinen Einblick in nationale wie auch internationale Forschungsergebnisse zu diesem Thema. Im letzten Kapitel fasse ich zusammen und gebe mein Fazit wieder.
  \end{flushleft}

  \begin{flushleft}
    Viele der Themen konnte ich nur anrei�en, da es sonst den Umfang dieser Arbeit �berschritten h�tte. Trotz allem hoffe ich, dass diese Hausarbeit dem Leser einen kleinen Einblick geben wird, welchen Einfluss die Qualifikation der Kita-Leitung auf die fr�hkindlichen Bildung in Kindertageseinrichtungen hat.
  \end{flushleft}

  \begin{flushleft}
    Des Weiteren m�chte ich darauf hinweisen, dass die Arbeit sich haupts�chlich auf Kita-Leitungen bezieht die freigestellt sind. Au�erdem verwende ich, der Einfachheithalber, immer die weibliche Form wenn z.B. von Mitarbeitern und Mitarbeiterinne oder �hnliches die Rede ist.
  \end{flushleft}

  \newpage


%---------------------------------------------------------------------------------------------------
% Der zweite Teil der Arbeit:
%---------------------------------------------------------------------------------------------------
\typeout{===== File: HAUPTTEIL}
%---------------------------------------------------------------------------------------------------
% Hauptteil
%---------------------------------------------------------------------------------------------------
%\newpage
%%\part{Hauptteil}

\chapter{Ein unklares Berufsprofil}\label{sec:2}
  \section{Anforderungen, Aufgaben und T�tigkeiten von Kita-Leitungen}\label{sec:2_1}

    \begin{flushleft}
      Kindertageseinrichtungen haben eine Vielzahl an Aufgaben zu bew�ltigen, doch die Hauptaufgabe von diesen Institutionen ist es Kinder zu bilden, betreuen und erziehen. \citep[vgl.][S.~9]{LeitungvKi2014} Gerade durch den steigenden gesellschaftlichen Anspruch - mit dazubeigetragen hat die PISA Studie - spielt die Qualit�t der fr�hkindlichen Bildung immer mehr eine Rolle. Die Frage \enquote{Wie die Qualit�t in den Kitas erh�ht werden kann?} stand zur Debatte. L�sungen f�r diese neue Herausforderung waren unteranderem die H�cherqualifizierung der Fachkr�fte und die Professionalisierung der p�dagogischen Arbeit. Mit dazu beitragen sollen die neuen kindheitsp�dagogischen Studieng�nge.
Im Zuge des ver�nderten Blick auf fr�hkindliche Bildung, wurde der Kita-Leitung immer mehr eine entscheidende Rolle f�r eine \enquote{gute und gesunde} Kita zugeschrieben. \citep[vgl.][S.~14]{Lupe2014} So wei"s man, dass Kita-Leitungen ein gro"ses Aufgebenspektrum abzudecken haben, sie m�ssen sowohl die Herausforderungen von kleinen bis mittleren Unternehmen meistern, als auch dem Kernauftrag von Kindertageseinrichtung, (bilden, erziehen und betreuen) gerecht werden.\citep[vgl.][S.~8]{LeitungvKi2014} Doch ein einheitliches l�nder- und tr�ger�bergreifendes Profil f�r Leitungskr�fte von Kindertageseinrichtungen existiert bis heute nicht. So ist wissenschaftlich weder nachzuvollziehen welche Kompetenzen und Qualifikationen Leitungskr�fte mitbringen sollten, noch welche Aufgaben sie zu bew�ltigen haben.\citep[vgl.][S.~14f]{Lupe2014} Die Klarheit die dieser Berufszweig dringend braucht, ist durch die deutschlandweite Vielfalt im Bereich der Kita-Landschaft schwerer zu erreichen. So setzt sich jede Kindertageseinrichtung aus verschieden Bausteinen zusammen, die abh�ngig von der Kita, eine andere Gewichtung haben. Jede Leitung arbeitet mit Eltern, Kindern und Mitarbeiterinnen zusammen. Hierbei gibt es nat�rlich gro"se Unterschiede. Doch sind es gerade die Bedingungen die jede Kita einzigartig macht. Zum einen ist es das jeweilige Landesrecht an dem sich Leitungskr�fte orientieren m�ssen, sowie die Tr�gerschaft, der Sozialraum, die Organisationsstruktur, die Organisationskultur und die Gr�"se der Organisation.(ebd) Gerade die Tr�gerschaft spielt eine gro"se Rolle bei der Verteilung der Aufgaben, die Kita-Leitungen zu bew�ltigen haben. Es gibt verschiedenste Konstellationen wie eine Kita geleitet werden kann. So kann es sein, das Verwaltungsaufgaben der Kita-Leitung �berlassen werden, f�r diese Aufgaben kann sich aber auch der Tr�ger selbst k�mmern. Das Gleiche gilt auch f�r den Punkt des p�dagogischen Konzepts. Folglich l�sst sich feststellen, dass sich somit auch die Anforderungen, der Handlungsspielraum und die Arbeitsbelastung in einem hohen Ma"s  voneinander unterscheiden k�nnen.\citep[vgl.][S.~9]{LeitungvKi2014} Hinzukommt das deutschlandweit immer noch 57\% der Leitungskr�fte nicht f�r die Leitungst�tigkeit freigestellt sind. So �bernehmen sie weiterhin T�tigkeiten in anderen Bereichen, wie etwa im Gruppendienst. Hier �bernehmen sie gruppen�bergreifende T�tigkeiten, arbeiten als Zweit- oder Erg�nzungskraft oder sind noch als Gruppenleitung t�tig. \citep[vgl.][S.~15]{Lupe2014}
Auch wenn die Anforderungen, Aufgaben und T�tigkeiten in der Gewichtung sehr variieren k�nnen, wird im folgenden versucht darzustellen welchem Spektrum eine Kita-Leitungen gerecht werden muss.
    \end{flushleft}


  \subsection{Der struktureller Rahmen der Leitungst�tigkeit}\label{sec:2_1_1}
    \begin{flushleft}
      Kitas unterstehen in den meisten Bundesl�ndern der Kinder und Jugendhilfe. Wie bereits erw�hnt (siehe 1.1) gibt es auf Bundesebene keine einheitlichen Rahmenbedingungen f�r Kindertageseinrichtungen. So tragen das jeweilige Landesrecht, die unterschiedlichen Bildungspl�ne oder -programme, Finanzierungssysteme der Bundesl�nder, Tr�ger und die Qualit�tsstandards  dazubei, dass eine un�bersichtliche Vielfalt in diesem Bereich herrscht.
Ein gro"ser Unterschied stellen in vielerlei Hinsicht die Finanzierungssysteme der einzelnen L�nder da.\citep[vgl.][S.~9]{LeitungvKi2014} Ein Vergleich  der Finanzierungssysteme zwischen den Nachbarl�ndern Hamburg und Schleswig-Holstein zeigt deutlich wie heterogen die Arbeit von Kita - Leitungen ist. Hamburg als Stadtstaat hat seit 2003 ein Subjektfinanzierungssystem. Wohingegen Schleswig-Holstein, wie die meisten Bundesl�nder in Deutschland, ein Objektfinanzierungssystem hat.\citep[vgl.]{Finanz2015} Doch was bedeutet das f�r die Leitungskr�fte von Kindertageseinrichtungen? Leitungskr�fte die in Schleswig-Holstein arbeiten haben eine gewisse Planungssicherheit, da ihre Kindertagesst�tten grundfinanziert werden. So m�ssen sich Kita-Leitungen nicht st�ndig mit Unsicherheiten bez�glich ihres Personals auseinandersetzen und k�nne ihnen, den Kindern und Eltern eine gewisse Kontinuit�t und Verl�sslichkeit bieten. Die Grundfinanzierung erm�glicht es den Kitas selbst zu w�hlen wer ihre Einrichtung besuchen darf. Wohingegen beim Subjekt finanzierten System der Nachfrager in diesem Fall die Eltern und Kinder finanziert werden. Tr�ger und Kita- Leitungen stehen hier vor der Herausforderung ihre Institutionen so attraktiv zu gestalten, dass alle ihre Pl�tze auch besetzt werden. Da kann es passieren, dass einige Angebotsstrukturen beim Personal nicht immer auf Gegenliebe st�"st. Hier ist die Leitung gefragt ihrem Personal eine gewisse Sicherheit zu vermitteln. So gleichen Hamburger Kitas in vielerlei Hinsicht mehr einem wirtschaftlichem Unternehmen.\citep[vgl.][S.~43]{Gutschein2009} Doch welche Aufgaben, besonders im Managementbereich, eine Kita-Leitung schlussendlich �bernimmt, h�ngt dann aber stark vom Tr�ger der jeweiligen Kita ab. Manche Leitungskr�fte und ihr Teams bekommen einen recht gro"sen Spielraum und gestalten ihre Konzepte und Richtlinien selber. Andere wiederum bekommen vom Tr�ger schon Richtlinien vorgegeben. Auch die Verteilung von Verwaltungsaufgaben ist sehr unterschiedlich. Manche Tr�ger �bernehmen selbst Verwaltungsaufgaben, wieder andere delegieren einige oder alle Verwaltungsaufgaben an die Kita-Leitung.\citep[vgl.][S.~9]{LeitungvKi2014} Weiterer entscheidende Punkte die den strukturellen Rahmen von Leitungskr�ften beeinflussen, sind das Personal selbst (denn so heterogen die Kita-Landschaft ist, so heterogen ist auch die Ausbildung der P�dagogen) und politische Gegebenheiten (Der Ausbau der unter drei J�hrigen Betreuung).(ebd) All diese Punkte zeigen die Komplexit�t des Aufgabenprofils von Leitungskr�ften. Im Folgenden wird versucht darzustellen wie dieses Aufgabenprofil aussieht.
    \end{flushleft}

  \subsection{Als Kita-Leitungen organisieren und managen}\label{sec:2_1_2}
    \begin{flushleft}
      Als Kita-Leitung ist es immer wichtig Management und Organisation im p�dagogischen Kontext zu betrachten. Durch den Prozess des Managen und Organisieren soll die Qualit�t der p�dagogischen Arbeit erh�ht werden. Die Leitungskraft sollte hierbei die zentrale Aufgabe spielen. Denn sie hat die Aufgabe Lerngelegenheiten und -anl�sse zu schaffen, sowohl f�r die Kinder wie auch f�r das p�dagogische Fachpersonal. Des Weiteren ist sie pr�gend f�r den Stil, Kultur und die Weiterentwicklung p�dagogischer Prozesse. \citep[vgl.][S.~11f]{LeitungvKi2014} Wenn man es ganz banal nimmt \enquote{ohne sie l�uft nichts}.
    \end{flushleft}

    \begin{flushleft}
      Um erfolgreich eine Kindertageseinrichtung f�hren und managen zu k�nnen, sollte man laut Simsa/Patak 2008 die sieben F�hrungsaufgaben beachten:
      \begin{enumerate}
        \item Mich selbst f�hren (Selbstf�hrung);
        \item Meine Mitarbeiterf�hren (Mitarbeiterf�hrung);
        \item Die Zusammenarbeit gestalten (Teamentwicklung, Elternarbeit);
        \item Die Organisation entwickeln (Organisationsentwicklung);
        \item Aufgabe und Ziele erf�llen;
        \item Den strategischer Rahmen f�r F�hrungsaktivit�ten setzen (F�hrungsstrategie)
        \item Das Umfeld beobachten, relevante Trends erkennen und Rahmenbedingungen wahrnehmen und deren Bedeutung f�r die Verantwortungsbereich einsch�tzen. \citep[vgl.][S.~41]{LeadershipNon2008}
      \end{enumerate}
    \end{flushleft}

    \begin{flushleft}
      Das Thema \textbf{Selbstf�hrung} ist sehr komplex und umfasst mehrere Punkte sowie Arbeitsorganisation, Fachkompetenz, Selbstkl�rung und Selbstreflexion, Selbstsorganisation, Karriereplanung, Arbeitsorganisation und Zeitmanagement, Stress- und Krisenmanagement und die eigene Haltung. \citep[vgl.][S.~12f]{LeitungvKi2014} So ist Selbstf�hrung der entscheidende Schritt um andere f�hren zu k�nnen. Wer sich selbst hinterfragt kann sich nur so weiterentwickeln und als F�hrungsperson glaubw�rdig agieren. Da die Aufgaben der Leitung mit gro"sen Verantwortungen verbunden sind, ist eine hohe pers�nliche Reife f�r dieses Arbeitsfeld von N�ten. So beeinflusst die Kita-Leitung mit ihrem eignen Handeln und Verhalten die Lebenswelt, Arbeitszufriedenheit und Gesundheit ihrer Mitarbeiterinnen. \citep[vgl.][S.~42]{LeadershipNon2008} Des Weiteren beeinflusst sie durch ihre Kompetenzen die Qualit�t der p�dagogischen Arbeit. Ganz banal gesagt, eine Kita spiegelt die F�hrungsqualit�t der Leitung wider.
    \end{flushleft}

    \begin{flushleft}
      \textbf{Mitarbeiterf�hrung} f�ngt mit der Auswahl des Personlas an. Hierbei ist entscheidend zu gucken wer passt ins Team und gibt es f�r ihn entsprechende Arbeitsbereiche. \citep[vgl.][S.~43]{LeadershipNon2008} Wenn neue Kollegen ins Team kommen, sollte die Kita-Leitung die Einarbeitung schon geplant haben. �bernimmt sie selber die Einarbeitung oder eine der anderen Mitarbeiterinnen.\citep[vgl.][S.~19]{Mitarbf�h2011} Des Weiteren ist das Binden der Mitarbeiterinnen an das \enquote{Unternehmen} (Kinder, Eltern, Team und Tr�ger) f�r eine gute Personalf�hrung unerl�sslich. Dieser Schritt ist besonders wichtig wenn eine hohe Fluktuation vermieden werden soll. Da es durch eine hohe Fluktuation schwer ist die angestrebte Qualit�t beizubehalten. Zudem m�ssen sich das Team, die Kinder und Eltern immer wieder auf neue Leute und Pers�nlichkeiten einstellen. Die Bindung an die Kita kann die Kita-Leitung unter anderem durch ihr eigenes Verhalten erreichen. Es gibt zwei entscheidende Werte die bei jedem Handeln dem Mitarbeiterinnen gegen�ber oberste Priorit�t haben: Wertsch�tzung und Fairness. Au"serdem sollte die Kita-Leitung sich als Berater und Begleiter ihres Personals verstehen. Hierf�r muss sie gegeben Falls verschiedenen Rollen (Starthelferin, Beraterin, Moderatorin, Gutachterin, Prozessbegleiterin, Trainerin, Therapeutin, Betreuerin, Sponsorin oder Patin) einnehmen.\citep[vgl.][S.~21ff]{Mitarbf�h2011} Als Kita-Leitung sollte man daher ein Gesp�r und die F�higkeit haben, andere und ihre Potenziale und Grenzen zu erkennen. Um sie dementsprechend zu f�rdern, zu motivieren und zu fordern.\citep[vgl.][S.~43]{LeadershipNon2008} Bei dem Thema Mitarbeiterf�hrung, spielt auch die Trennung von Mitarbeiterinnen eine Rolle. Die Kita-Leitung fungiert hier in einer unliebsamen Rolle. Hierbei sollte ihr bewusst sein, dass ein einf�hlsamer Umgang sehr wichtig ist.\citep[vgl.][S.~37]{Mitarbf�h2011}
    \end{flushleft}

    \begin{flushleft}
      Qualit�t und ein gut funktionierendes Team k�nnen gerade in der Kindertagesbetreuung nicht getrennt von einander betrachtet werden. Nur weil jeder Einzelne hervorragende Kompetenzen besitzt, hei"st dies noch lange nicht, dass sie auch gut im Team zusammenspielen k�nnen. Daher sollte jeder Kita-Leitung das Thema \textbf{Teamentwicklung} sehr am Herzen liegen.\citep[vgl.][S.~44]{LeadershipNon2008} Doch arbeitet die Kita-Leitung nicht nur mit ihrem Team zusammen, sondern auch mit den Eltern, dem Tr�ger und nat�rlich auch mit den Kindern. Au"serdem vernetzt sie sich mit anderen Einrichtung im Sozialraum.\citep[vgl.][S.~13]{LeitungvKi2014} Da sie hier mit vielen Parteien \textbf{zusammen arbeitet} ist die richtige Beobachtung, Vermittlung und  Kommunikation ein wichtiges und entscheidendes Gut. \citep[vgl.][S.~44]{LeadershipNon2008}
    \end{flushleft}

    \begin{flushleft}
      Bei der \textbf{Organisationsentwicklung} geht es unter anderem um Regeln, Strukturen und Prozessbeschreibungen.\citep[vgl.][S.~45]{LeadershipNon2008} So wird die Arbeit der Kita-Leitung, bei dem Thema Organisationskultur, vom Tr�ger gepr�gt. Erst einmal gibt der Tr�ger die Regeln, Normen und Werte f�r den t�glichen Umgang von Gro"s und Klein vor. So sind das Bild vom Kind und das allgemeine Menschenbild eine guter Wegweiser f�r alle und bestimmen die jeweilige Kultur jeder einzelnen Kita. Die Organisationskultur und das Organisationsklima gehen Hand in Hand �ber. Die Atmosph�re, die stark von den jeweiligen Akteurinnen abh�ngig ist, spiegelt den Umgang von Gro"s und Klein wider. Die Kita-Leitung kann durch ihr F�hren und Leiten, dies stark beeinflussen. Auch die gezielte Steuerung der Organisation, ist eine Aufgabe die die Kita-Leitung zu erf�llen hat. Dies muss sie tun um die Qualit�t ihrer Einrichtung gew�hrleisten zu k�nnen. Teil dieser Aufgabe ist es, Ablaufe nach ihrer Qualit�t und ihrer Umsetzung zu beurteilen. Wenn n�tig schreitet sie ein, um das gew�nschte Ziel zu erreichen. Ein ebenfalls sehr komplexes Aufgabengebiet sind die Organisationsentwicklungsprozesse. Sie bringen die Kita im \enquote{wahrsten Sinne des Wortes} ein St�ck nach vorne. Der Weiterentwicklungsprozess kann sehr unterschiedlich beeinflusst sein und kann dadurch nat�rlich auch sehr unterschiedlich aussehen. So kann der Prozess sowohl von eignen Zielen wie auch von �u"seren Anforderungen ins Rollen gebracht werden. Die Weiterentwicklung sollte sehr strukturiert ablaufen. Dies sollte die Kita-Leitung sich klar machen und sich unter anderem die folgenden Fragen stellen \enquote{Was habe ich?}, \enquote{Was will ich erreichen?}, \enquote{Was brauche ich um dieses Ziel zu erreichen?} und \enquote{In welchem Zeitraum schaffe ich das?}. Da der Prozess der Ver�nderung nicht immer Reibungslos abl�uft, kann die Begleitung von einer au"senstehenden Person sehr hilfreich sein. Gerade beim Thema Teamentwicklung ist es angebracht jemanden von au"sen einzuladen, um Teamprozesse voran zu treiben \citep[vgl.][S.~19]{LeitungvKi2014}
    \end{flushleft}

    \begin{flushleft}
      An Kindertagesst�tten werden viele Erwartungen und Aufgabe von au�en herangetragen. So werden Aufgaben und Auflagen vom Gesetzgeber und von Tr�gern an eine Kita weitergeleitet. Des Weiteren haben Eltern und die Gesellschaft im allgemeinen eine gewisse Erwartungshaltung, was eine Kita zu leisten hat. \citep[vgl.][S.~45f]{LeadershipNon2008} Aufgaben sind Auftr�ge f�r Leis�tungen, die vom Personal im Hinblick auf die Ziele der Organisation zu erf�llen sind.\citep[zitat][S.~14]{LeitungvKi2014} Daraus l�sst sich schlie�en, dass unterschiedliche Akteurinnen unterschiedliche \textbf{Aufgaben und Ziele zu erf�llen} haben. Das p�dagogische Fachpersonal ben�tigt sowohl fachliche, wie auch soziale Kompetenzen im Umgang mit Kindern, Eltern und ihren Teammitgliedern. Die Kita-Leitung selber hat, wie bereits mehrfach erw�hnt, ein sehr komplexes Aufgabenfeld. Zu ihren Aufgaben geh�rt es unter anderem Managementaufgaben, sowie betriebwirtschaftliche Prozesse zu erf�llen. Au�erdem hat sie die Aufgabe die an die Kita herangetragenen Erwartungen, Aufgaben und Auflagen f�r ihre Kita zu �bersetzen und �bertragbar zu machen. Dabei stellt sie sicher, dass fachliche Standards, Aufgaben und Ziele erf�llt werden. Hierzu geh�ren sowohl inhaltlich - konzeptionelle Dinge, wie auch die Umsetzung des p�dagogisch Konzepts.\citep[vgl.][S.~14]{LeitungvKi2014}
    \end{flushleft}

    \begin{flushleft}
      Die \textbf{F�hrungsstrategie} ist ein Aufgabenbereich in dem festgelegt wird, welchen Weg die Kindertagesst�tte gehen soll. Hier werden Konzepte und Konzeptionen entwickelt, in denen festgelegt wird welche Leitbilder, Visionen, Strategien, Werte und Normen f�r diese Institution gelten. Dies soll allen Beteiligten den t�glichen Umgang erleichtern. Da alle Mitarbeiterinnen eine klare Orientierung f�r ihren Arbeitsalltag bekommen. Wenn dies gut geplant ist, ist das eigentlich F�hren an diese Stelle nicht mehr von N�ten. Die Kita-Leitung fungiert dann eher als Richtungsweisende und nicht mehr als F�hrung im eigentlichen Sinne.\citep[vgl.][S.~46]{LeadershipNon2008}
    \end{flushleft}

    \begin{flushleft}
      Unerl�sslich ist es als F�hrungskraft einer Kita \textbf{das Umfeld zu beobachten, relevante Trends zu erkennen und Rahmenbedingungen wahrzunehmen und deren Bedeutung ihren Verantwortungsbereich einzusch�tzen}. In Simsa und Patak (2008) findet man zwar eine unvollst�ndige, aber trotz dessen recht hilfreiche Aufz�hlung, was eine Leitungskraft, nicht aus dem Auge verlieren sollte. Des Weiteren ist auch die Zusammenfassung dieses Aufgabenbereiches von Ulber/Strehmel (2014) sehr Hilfreich. So sollte eine Kita-Leitung den Personalmarkt, die Entwicklung in der eigenen Kindertagesst�tte, die gesetzliche Lage  relevante technologische Entwicklungen, relevante Trends, gesellschaftliche Entwicklungen, die Rahmenbedingungen ehrenamtlicher T�tigkeiten, Entwicklungen im Umfeld sowie in anderen Kindertagesbetreuungen und gesetzliche Richtlinien und Rahmungen zu beobachten und zu interpretieren. \citep[vgl.][S.~47]{LeadershipNon2008} \citep[vgl.][S.~21]{LeitungvKi2014}
    \end{flushleft}

    \begin{flushleft}
      Um dem Spektrum an Anforderungen und Aufgaben gerecht zu werden, ist die Kooperation und Vernetzung mit anderen Einrichtungen und Institutionen f�r Kita-Leitungen unerl�sslich. Diese erhalten den Informationsfluss den sie ben�tigen und k�nnen diese Quellen als Partner und St�tze in der Bildung, Erziehung und Betreuung von Kindern nutzen.\citep[vgl.][S.~21f]{LeitungvKi2014}
    \end{flushleft}

    \newpage

\newpage

\chapter{Das Qualifikationsgef�ge des Leitungspersonals}\label{sec:3}
  \section{Qualifikationsformen von Leitungskr�ften}\label{sec:3_1}
    \begin{flushleft}
      Der Ort an dem F�hrungskr�fte von Kindertageseinrichtungen ausgebildet werden, ist heute noch meist die Fachschule f�r Sozialp�dagogik.\citep [vgl.] [S.~95]{EmpirB2014} So habe 20,4\% der zeitlich freigestellten Leitungskr�fte einen einschl�gigen Hochschulabschluss. Der Gro"steil 77\%  der Kita-Leitungen sind Erzieherinnen. \citep[vgl.][S.~35]{Laender2013} Des Weiteren �bernimmt 57\% des Leitungspersonals weitere Aufgaben unter anderem im Gruppendienst.
\enquote{\textit{Eine p�dagogische Fachkraft soll nicht die p�dagogische Arbeit mit den Kindern vernachl�ssigen m�ssen, weil sie Leitungsaufgaben wahrnimmt}}.\citep[zitat][S.~5]{Laender2013}
Wie bereits in 1.1 erw�hnt haben Leitungskr�fte einen komplexen Aufgabenbereich, der sich im Zuge der Zeit gewandelt hat. So galten Leitungst�tigkeiten lange als nebens�chlich und aus diesem Grund wurde ihnen keine rechte Aufmerksamkeit geschenkt. Erst seit dem die Fach�ffentlichkeit der Kita-Leitung eine Schl�sselposition f�r die Qualit�t einer Kita zuschreibt, ist dieser Berufszweig mehr und mehr in den Fokus ger�ckt worden. Durch das hohe Anforderungsprofil, dem die Leitungkr�fte ausgesetzt sind, r�ckte die Frage nach dem Wissens- und Kompetenzerwerbs immer st�rker in den Vordergrund. Des Weiteren wirkten gesellschaftspolitische Geschehen, wie etwa die Pisa-Studie, der Bolongna-Prozess und der Europ�ische und Deutsche Qualifikationsrahmen f�r lebenslanges Lernen, auf die Entwicklung in diesem Berufszweig aus.\citep [vgl.] [S.~95f]{EmpirB2014} Im Zuge dessen wurde auch die Ausbildung der p�dagogischen Fachkr�fte und das Qualifikationsniveau der Kita-Leitungen st�rker ins Visier genommen.\citep[vgl.][S.~34]{Laender2013} Da das Qualifikationsniveau von Leitungskr�ften sehr unterschiedlich ist und nicht klar gesagt werden kann welcher Ausbildungszweig das Leitungspersonal am besten mit dem Wissen und den Kompetenzen ausstattet die ben�tigt werden um eine Kita leiten zu k�nnen, wird im folgenden geguckt und verglichen wie die Lehrpl�ne der verschiedenen Schul- und Weiterbildungsformen aussehen.
    \end{flushleft}

  \newpage
  \section{Die Ausbildung an Fachschule f�r Sozialp�dagogik}\label{sec:3_2}
    \begin{flushleft}
      Wie bereits in Abschnitt \ref{sec:3_1} erw�hnt- haben 77\% der zeitlich freigestellten Leitungskr�fte einen Fachschulabschluss. \citep[vgl.][S.~34]{Laender2013} Dieser kann in vier unterschiedlichen Formen erreicht werden. Die mit 87\% am h�ufigsten besuchte Form impliziert eine zweij�hrige theoretischen Teil und einen einj�hrigen praktischen Teil. Eine der anderen Formen ist der zeitlich verl�ngerter qualifizierende Bildungsgang, hier erlangen die angehenden Erzieherinnen, nach vier Jahren, zudem noch das Abitur.\citep[vgl.][S.~12]{AusbildEri2012} Des Weiteren besteht die M�glichkeit an einigen Schulen die Ausbildung in Teilzeitform zu absolvieren. Die Dauer dieser Form betr�gt insgesamt vier bis f�nf Jahre. Hier von sind drei Jahre fachtheoretisch und ein bis zwei Jahre sind praktisch angelegt.\citep[vgl.][]{Berufsbe2015} Es besteht auch die Option der berufsbegleitenden Ausbildung. Diese Ausbildungsform richtet sich an Menschen die schon im sozialen Bereich t�tig sind, aber keine abgeschlossene Berufsausbildung in diesem Bereich haben. Die Ausbildung erfolgt z.B. in Hamburg drei Jahre und die angehenden Erzieherinnen sind dazu verpflichtet einer T�tigkeit in einem erzieherischen bzw. sozialp�dagogischen Arbeitsfeld von mindestens 15 Stunden in der Woche nachzugehen.\citep[vgl.][]{Fach2015} So heterogen wie die Ausbildungform der Erzieherin ist, so heterogen sind auch die Ausbildungsmodelle an den Fachschulen. So herrscht an den Fachschulen in Baden-W�rttemberg, Bayern, Bremen, Hessen, Nordrhein-Westphalen, Rheinland-Pfalz, Saarland und an einigen Schulen in Sachen-Anhalt ein additives Ausbildungsmodell. Die angehenden Erzieherinnen erlangen ihren Abschluss nach einem zweij�hrigen fachtheoretischen Teil und dem anschlie"senden einj�hrigen Anerkennungsjahr in der Praxis. Dieses Modell ist mit 68\% das in Deutschland meist praktizierteste. Ebenfalls besteht die M�glichkeit einer integrativen Ausbildung. Hier wechseln sich der schulische und der praktische Teil w�hrend der Ausbildung ab. Dieses Ausbildungsmodell wird in Brandenburg, Berlin, Hamburg, Mecklenburg-Vorpommern, Niedersachen, Sachsen, Schleswig-Holstein und teilweise in Sachsen-Anhalt praktiziert. Nur in Th�ringen herrscht ein integratives Ausbildungsmodell mit halbj�hrigen Berufspraktikum.\citep[vgl.][S.~19f]{AusbildEri2012} Da es so ein gro"ses Spektrum an Ausbildungsformen und  Ausbildungsmodellen gibt, werden im Folgenden die  Rahmenlehrpl�ne der Bundesl�nder verglichen. So ist festzustellen, dass auch die Ausbildungsinhalte deutschlandweit sehr heterogen sind. Die Inhalte werden von den Verordnungen der Kulturministerkonferenz und den Lehrpl�nen der L�nder vorgegeben.
    \end{flushleft}

    \newpage
    \begin{flushleft}
      Die Rahmenvereinbarung legt folgende Inhaltsbereiche fest:
      \begin{itemize}
        \item[-] Kommunikation und Gesellschaft
        \item[-] Sozialp�dagogische Theorie und Praxis
        \item[-] Musisch-kreative Gestaltung
        \item[-] �kologie und Gesundheit
        \item[-] Organisation, Recht und Verwaltung
        \item[-] Religion / Ethik nach dem Recht der L�nder.
      \end{itemize}
    \end{flushleft}

    \begin{flushleft}
      Die Verordungen l�sst den Schulen einen gro"sen Gestaltungsspielraum, dies f�hrt zu unterschiedlichen Profilen und Schwerpunkten in der Ausbildung. Des Weiteren spielen die Bildungspl�ne, Bildungsprogramme bzw. Bildungsleitlinien der fr�hkindlichen Bildung eine gro"se Rolle, da diese ebenfalls Anforderungen an die P�dagogen stellen. So gibt es bei den 16 Bundesl�ndern auch 16 verschiedene Bildungspl�ne. \citep[vgl.][S.~12ff]{AnfEr2011 }Trotz der Unterschiedlichkeiten bei den Ausbildungsinhalten kann gesagt werden, dass obwohl die Ausbildung als Breitbandausbildung ausgelegt ist, sich die angehenden Erzieherinnen haupts�chlich mit der fr�hkindlichen Bildung auseinandersetzen.\citep[vgl.][S.~22]{AusbildEri2012} Inwieweit explizit auf Leitungst�tigkeiten eingegangen wird, kann leider nicht gesagt werden.
Die Ausbildung zur Erzieherin hat die Aufgabe ein weites Spektrum an Themen abzudecken, um am Ende \enquote{multifunktionale Fachkr�fte} zu bekommen. Daher kann man die Ausbildung als Pflicht sehen in der viele Themen angeschnitten werden. Die anschlie"sende Spezialisierung, durch Fort- und Weiterbildungen, ist dann die eigentliche K�r. So haben auch die befragten Fachschul- und Abteilungsleitungen, die an der qualitativen Befragung zum Thema \enquote{Anforderungen an die Ausbildung von Erzieherinnen und Erziehern} teilgenommen haben, angegeben dass Fort- und Weiterbildungen f�r diesen Berufszweig unerl�sslich sind. Daher wird im Abschnitt \ref{sec:3_3} auf das Thema Fort- und Weiterbildungen eingegangen.\citep[vgl.][S.~41]{AnfEr2011}
    \end{flushleft}


  \section{Professionalisierung durch Akademisierung}\label{sec:3_3}
    \begin{flushleft}
      Wenn es um das Thema Professionalisierung von Leitungskr�ften geht, wird h�ufig gefordert,
      dass Kita-Leitungen einen einschl�gigen Hochschulabschluss vorweisen sollten.\cite[vgl.][S.~267]{Ausblick2014}
      So haben aber im Durchschnitt deutschlandweit 20,4\% der freigestellten Kita-Leitungen einen einschl�gigen
      Hochschulabschluss. Dies ist im Vergleich zu anderen Bereichen der Kinder- und Jugendhilfe sehr gering.
      Trotz des Wunsches nach Kita-Leitungen mit Hochschulabschluss, kann nicht gesagt werden dass eine h�heres
      Qualifizierungniveau besser auf die Aufgaben einer Kita-Leitung vorbereitet. \citep[vgl.][S.~34]{Laender2013}
      Im Folgenden soll betrachtet werden welche einschl�gigen Hochschulabschl�sse Kita-Leitungen haben und welches
      Studium die Kompetenzen vermittelt die eine Kita-Leitung ben�tigt. Hierf�r sollen sich exemplarisch,
      die Studieng�nge in Hamburg (eine Ausnahme ist der Studiengang Heilp�dagogik, dieser ist nicht Hamburg vorhanden)
      angeschaut werden, wo unter anderem mit 46\% (in Bremen 47\%) mit die meisten freigestellten Kita-Leitungen mit
      einschl�gigem Hochschulabschluss t�tig sind. So haben  Kita-Leitungen mit einschl�gigen Hochschulabschl�ssen
      meistens einen Abschluss in Sozialer Arbeit oder in Sozialp�dagogik (74\%), den sie an eine Fachhochschule
      erlangt haben. Absolventinnen die einen erziehungswissenschaftlichen oder sozialp�da�gogischen
      Universit�renabschluss haben, sind mit 20,5\% nicht so stark vertreten. Heilp�dagoginnen und Kindheitsp�dagoginnen
      (mit und ohne Master) sind dagegen mit 2,6\% und 0,3\% unterdurchschnittlich vertreten. Bei dem Berufzweig der
      Kindheitsp�dagoginnen kann es auch daran liegen, dass dieser Studiengang noch recht jung ist.
      Die Absolventinnen dieses Studienganges versuchen sich erst seit kurzem in diesem Berufszweig zu platzieren.
      Auch das Alter bzw. die Erfahrung spielen hier ebenfalls eine entscheidende Rolle. So sind 45\% der Kita-Leitungen
      zwischen 50 und 60 Jahre alt. \citep[vgl.][S.~101ff]{EmpirB2014} Es besteht noch die Frage inwieweit die
      Studieng�nge auf die sp�tere Leitungsposition vorbereiten. Aus diesem Grund wird im Folgenden betrachtet welche
      F�cher (Seminare, Tutorien und Vorlesungen) die Studieng�nge anbieten. Das Studienfach Soziale Arbeit an der
      HAW Hamburg, vermittelt Wissen und Kompetenzen in den Bereichen Recht, Erziehungswissenschaft, Psychologie und
      Sozialwissenschaften. Des Weiteren k�nnen die Studierenden aus diesen Schwerpunkte w�hlen: Gesundheit, Pr�vention
      und Rehabilitation, Existenzsicherung, Resozialisierung und Integration, Kinder-, Jugend- und Familienarbeit,
      Kultur-, Bildungs- und Stadtteilarbeit. Zu dem absolvieren die Studierenden im f�nften Semester ein Vollzeitpraktikum.\citep[vgl.][]{SozialH2015}
      Studierende bzw. zuk�nftige Kita-Leitungen die Erziehungs- und Bildungswissenschaften an der Universit�t Hamburg haben
      die Wahl sich auf einen von den drei Schwerpunkten festzulegen: Sozialp�dagogik/Kinder- und Jugendbildung, Erwachsenen-
      und Weiterbildung oder Behindertenp�dagogik -- Inklusion und Partizipation bei Behinderung und Benachteiligung festzulegen.
      Auf ihr Hauptfachbezogen erlangen sie dann in folgenden F�chern Kompetenzen: Grundlagen der Erziehungswissenschaft,
      Erziehungswissenschaftliche Forschungsmethoden, Psychische Bedingungen und Prozesse in Bildung und Erziehung und Geschichte,
      Theorien und gesellschaftliche Bedingungen von Erziehung, Bildung und Sozialisation.\citep[vgl.][]{Erziewis2015}
      Das Studienfach der Sonder- oder Heilp�dagogin, dessen Absolventeninnen recht selten als Kita-Leitung fungieren,
      ist in Hamburg nicht vertreten. Aus diesem Grund wird der Bachelorstudiengang Heilp�dagogik der Hochschule Hannover
      vorgestellt. Leitungskr�fte die einen Abschluss an dieser Hochschule anstreben, studieren die folgenden F�cher:
      Wissenschaftlich denken und professionell handeln, Erziehen und F�rdern, Menschliches Verhalten und Erleben erkl�ren
      und verstehen, Soziale Strukturen analysieren und beeinflussen, Diagnostizieren, planen und evaluieren, Beraten und
      Kooperieren, Begleiten und Partizipation erm�glichen, Erziehen und F�rdern, Heilp�dagogisches Praxisprojekt,
      Wissenschaft anwenden, Begleiten und Partizipation erm�glichen sowie \textbf{Leiten und Kooperieren}.\citep[vgl.][]{Heilp2015}
      Die Kompetenzen und das Wissen, welche im j�ngsten der vier Studieng�nge (Kindheitsp�dagogik) vermittelt wird,
      sehen wie folgt aus: es werden Grundlagen geschaffen in Erziehungs- und Bildungswissenschaften, Psychologie,
      Sozialwissenschaften und Recht sowie in empirischen Methoden. Zu dem erfolgt im viertem Semester eine Einf�hrung
      in die drei Studienschwerpunkte: Kompetenzentwicklung in der Kindheit, Familienberatung und Institutionsentwicklung / Management.
      Im f�nften und sechsten Semester entscheiden sich die Studierenden f�r zwei der drei Hauptf�cher.\citep[vgl.][]{Kindheitsp2015}
      Wenn man sich die vier Studieng�nge genau anguckt, haben nur zwei der Hochschulabschl�sse Management und Leiten als
      eigenst�ndiges Fach. Dies sind die Studieng�nge, die unterdurchschnittlich bei den Leitungskr�ften mit einschl�gigen
      Hochschulabschluss, vertreten sind. Zum einen der junge Studiengang der Kindheitsp�dagogik und dann noch der
      Studiengang der Heilp�dagogik. Es ist klar, dass dies Daten nicht repr�sentativ f�r Deutschland und ihre Kita-Leitungen sind.
      Denn wie bereits erw�hnt sind Kita-Leitungen meist zwischen 50 und 60 Jahre alt. Daraus kann geschlossen werden, dass ihr
      Studium sich von dem heutigem Studium unterscheidet. Alleine durch den Bologna-Prozess, fand ein gro"ser Wandel in der
      Studienlandschaft statt. Trotzdem kann gesagt werden, dass ein einschl�giger Hochschulabschluss alleine kein Garant ist,
      dass alle ben�tigen Kompetenzen wie zum Beispiel Management bzw. das Leiten von Bildungs-, Erziehungs- und Betreuungseinrichtungen
      Schwerpunkt in den einschl�gigen Studieng�ngen ist. Aus diesem Grund sind auch f�r Hochschulabsolventinnen Weiterbildungsm�"snahmen
      essenziell. \citep[vgl.][S.~269]{Ausblick2014}
    \end{flushleft}
  \newpage
  \section{Fort- und Weiterbildung im Leitungsgef�ge}\label{sec:3_4}
    \begin{flushleft}
      Fort- und Weiterbildungen sind aus der Kindertagesbetreuunung nicht wegzudenken. Die Ausbildung der p�dagogischen Fachkraft kann als das Fundament gesehen werden und die berufsbegleitende Weiterbildung baut darauf auf und erweitert gew�nschte Kompetenzen. Daher kann der Weiterbildung eine zentrale Rolle, bei der Professionalisierung des Berufsbildes der Kita-Leitung, zu geschrieben werden. \citep[vgl.][S.~268]{Ausblick2014} So ergab die Studie \enquote{Kinder und t�tige Personen in Tageseinrichtungen und in �ffentlich gef�rderter Kindertagespflege} des statistischen Bundesamts zwar, dass ein Gro"steil der Leitungen in Kindertageseinrichtungen eine Ausbildung zur Erzieherin hat. Doch das Ergebnis dieser Studie spiegelt nicht wider, wieviele von den 77\% eine Fort- oder Weiterbildung gemacht haben, um ihr Wissen und ihre Kompetenz zu erweitern.\citep[vgl.][S.~36]{Laender2013} Denn die bundesweite Befragung von Einrichtungsleitungen und Fachkr�ften zu den Themen Qualifikationen und Weiterbildung ergab, dass dieser Berufszweig sehr weiterbildungsaffin ist. So haben laut dieser Befragung 96,9\% der Leitungskr�fte an Weiterbildungsma"snahmen teilgenommen.\citep[vgl.][S.~34]{QualWei2012} So sind sich zwar 78\% der Befragten einig, dass die Aneignung neuer Kenntnisse und F�higkeiten sehr wichtig ist, doch sehen sie auch Verbesserungspotenzial in diesem Bereich. Besonders die Anerkennung von Weiterbildungsma"snahmen wird bem�ngelt. \citep[vgl.][S.~57]{QualWei2012}
    \end{flushleft}

    \begin{flushleft}
      Warum ist Weiterbildung f�r Kita-Leitungen �berhaupt so ein wichtiges Thema? Laut Schelle gibt es drei Gr�nde warum gerade f�r Kita-Leitungen das Thema Fort- und Weiterbildung ma�geblich ist. So k�nnen und sollten Kita-Leitungen diese R�ume nutzen um sich �ber neue Trends und tagespolitische Themen oder neue p�dagogische Ans�tze auszutauschen und diese dann f�r ihre Einrichtung �bertragbar machen. Des Weiteren spielen Weiterbildungsma�nahmen, gerade f�r den Managementbereich, eine Gro�e Rolle. Da die Kompetenzen die f�r diesen Bereich ben�tigt werden, kaum an Fachschulen vermittelt werden. Ein einschl�giger Hochschulabschluss alleine garantiert aber auch nicht, dass im Managementbereich Kompetenzen erworben wurden. Da bislang nur wenig Studieng�nge einen Schwerpunkt auf Management bzw. die Leitung von Institutionen der Bildung, Erziehung und Betreuung von Kindern legen. Au�erdem sollte bedacht werden, dass Kita-Leitungen meist ehemalige Mitarbeiterinnen des Teams waren und nun einen Rollenwechsel vollzogen haben. Sie sind jetzt nicht mehr nur Kollegin, sonder Leitung der Kita. Hier kann eine Weiterbildung allen Beteiligten helfen, den Rollenwechsel zu meistern.\citep[vgl.][S.~269]{Ausblick2014}
    \end{flushleft}

    \newpage





\newpage

\chapter{Empirische Untersuchungen zur Qualifikation der Kita-Leitung}\label{sec:4}
  \section{Nationale Forschungsergebnisse}\label{sec:4_1}
    \begin{flushleft}
      Einleitend muss erw�hnt werden, dass Kita-Leitungen und ihre Qualifikation und den daraus resultierenden Einfluss auf die fr�hkindlichen Bildung in Kindertageseinrichtungen, im deutschsprachigen Raum noch nicht ausreichend empirisch untersucht worden. Wie bereits in dieser Arbeit erw�hnt wird- wird  der Kita-Leitung und ihrer Schl�sselfunktion erst seit j�ngster Zeit mehr Aufmerksamkeit geschenkt. Daher bilden gerade internationale Studien den Kern der folgenden Untersuchung \enquote{Ob und inwieweit die Qualifikation der Kita-Leitung eine Rolle spielt?}. So wird unter anderem auf nationaler Ebene die amtliche Kinder- und Jugendhilfestatistik seit 2011 erfasst wie viele Personen Leitungst�tigkeiten �bernehmen und wie viele Kita-Leitungen f�r diesen Arbeitsbereich freigestellt sind und welche Abschl�sse in Leitungspositionen vertreten sind. Doch gibt diese Erhebung nicht wieder, ob und wie die Qualifikation der Leitungskraft sein sollte.\citep[vgl.][S.~33]{Laender2013} An Hand der vorherigen Kapitel wird einem zwar klar, dass der Arbeitsbereich der Kita-Leitung sehr komplex ist und viele Kompetenzen verlangt. Die Studie \enquote{Qualifikationen und Weiterbildung fr�hp�dagogischer Fachkr�fte} zeigt deutlich, dass Weiterbildungen einen wichtigen Teil bei der Professionalisierung dieser Berufssparte darstellt.\citep[vgl.][S.~57]{QualWei2012} Au"serdem zeigen Studien, dass die Qualifizierung im Sinne der Aus- und Weiterbildung zwar auch von den Akteurinnen verlangt wird, aber ihnen Berufserfahrung und damit einhergehend ein gewisses Alter der Kita-Leitung ebenfalls von Bedeutung sind.\citep[vgl.][S.~130]{EmpirB2014} Doch gibt, wie bereits erw�hnt, keine nationale Studien wieder, ob das Qualifikationsniveau von Kita-Leitungen und die Qualit�t fr�hkindliche Bildung in Kindertageseinrichtungen miteinander korrelieren. Aus diesem Grund wird in Abschnitt \ref{sec:4_2} auf internationale Studien und Forschungsergebnisse eingegangen und geguckt inwieweit das Thema international erforscht wurde.
    \end{flushleft}
  \newpage
  \section{Internationale Forschungsergebnisse}\label{sec:4_2}
    \begin{flushleft}
      Die Einfl�sse auf die Qualit�t fr�hkindliche Bildung ist auf internationaler Ebene schon l�nger ein Thema,
      gerade in Gro"sbritannien wird dieses Thema schon l�nger untersucht. Aus diesem Grund kommen die meisten
      Forschungsergebnisse hierher. Bei diesen Untersuchungen wurde ebenfalls erhoben, welchen Einfluss die
      Qualifikation der Kita-Leitung auf die fr�hkindliche Bildung hat. Die bekanntesten internationalen
      Forschungsergebnisse im Bereich der fr�hkindlichen Bildung entstammen der EPPE (Effective Provision
      of Pre-school Education) Studie. Diese wurde zwischen 1997 und 2003 in England durchgef�hrt. An der Studie
      haben ca. 2800 Kinder aus 141 Kindertageseinrichtungen und 300 Kinder die keine Kindertageseinrichtung
      besucht haben teilgenommen. Im Laufe der Studie wurden Daten �ber die Kinder und ihre Eltern, sowie �ber
      deren Familienverh�ltnisse erhoben. Des Weiteren- und f�r diese Arbeit interessant- wurden ebenfalls Daten
      �ber das p�dagogische Personal und der Qualit�t der Kindertageseinrichtung erhoben.\citep[vgl.][]{ForschungERS2015}
      So zeigen die Ergebnisse der Studie, dass ein h�her qualifizierteres Personal auch f�r mehr Qualit�t in der
      Einrichtung steht. Dies gilt im Besonderen, wenn die Kita-Leitung eine h�here Qualifizierung besitz.\citep[vgl.][S.~4]{EPPE2004}
    \end{flushleft}
    \begin{flushleft}
      Auch weitere Studien aus England, wie zum Beispiel die dreij�hrige L�ngsschnittstudie zum Thema
      \textit{Kindheitsp�dagogen: Eine Untersuchung �ber die Entwicklung, die Leitungsposition und die Auswirkungen}
      \footnote{Early Years Professional Status arbeiten in England mit Kindern von 0 bis 5 Jahren. Das Studium wurde 2007 von der Regierung ins Leben gerufen und ist vergleichbar mit dem in Deutschland existierenden Studium f�r Kindheitsp�dagogen}
      (Eigene �bersetzung) \citep[vgl.][]{EYP2011} (\textit{Longitudinal Study of Early Years Professional Status: an exploration of progress, leadership and impact})
      aus 2009, die im Auftrag der Children's Workforce Development Council (CWDC) durchgef�hrt w�rde, spieglen
      wider dass die Qualifizierung und auch die Professionalisierung von Kita-Leitung einen erheblichen Einfluss
      auf die Qualit�t der fr�hkindlichen Bildung hat. Dies beruht haupts�chlich auf ihr Verst�ndnis von Qualit�t
      und wie sie diese erreichen k�nnen.\citep[vgl.][S.~86]{Long2012} Des Weiteren gibt es noch die ELEYS�-Studie
      (Effective Leadership in the Early Years Sector), hier hat ein Team anhand von Daten aus der EPPE-Studie,
      erhoben wie sich eine Kita-Leitung verhalten muss um eine hohe Qualit�t in der p�dagogischen Arbeit zu
      erreichen. Die Wissenschaftler arbeiteten heraus welche Punkte hierf�r von Bedeutung sind. Zum einen sind
      es die eigenen Kompetenzen und die der Mitarbeiterinnen miteinander zu verkn�pfen und in Einklang zu bring.
    \end{flushleft}
    \begin{flushleft}
      Ebenfalls ist es wichtig neue Trends zu erkennen und sie mit dem Altbekannten zu verkn�pfen, sowie eine klare
      Definition der Rollen und deren Aufgabenbereiche. Die Ergebnisse zeigen ebenfalls, dass eine klare und
      gemeinsame Vorstellung vom Bild des Kindes und eine damit einhergehende p�dagogisches Konzept zu einer hohen
      Qualit�t in der p�dagogischen Arbeit beitr�gt. Au�erdem ist es wichtig die Teammitglieder zu ermutigen sich
      und ihre eigene Arbeit kritisch zu reflektieren. Der ganz entscheidende Punkt und der bei allen vorherigen
      Punkten greift, ist die Kommunikation mit und im Team.\citep[vgl.][S.~46]{LeitungvKi2014} Abgesehen von den
      oben genannten Ergebnissen, kam auch hier herraus, dass ein Kita-Leitung die durch ihre Ausbildung gut auf
      ihre Rolle als Leitungskraft vorbereitet wurde ein wichtige Element ist f�r die Qualit�t der fr�hkindlichen
      Bildung in Kindertageseinrichtungen. Besonders weil Teams heute immer gr��er werden und sie auch
      multi-professioneller werden und sein m�ssen.\citep[vgl.][S.~27]{EffeLead2006}
    \end{flushleft}
    \begin{flushleft}
      Eine Studie die ebenfalls aus Gro�britannien stammt, die 2013 von Carol Aubrey, Ray Godfrey und Alma Harris angelegt wurde, befasst
      sich mit der Bedeutung von Leitung in Einrichtungen der Fr�hen Bildung.\citep[vgl.][S.~45]{LeitungvKi2014}
      Au�erdem f�hrte das Team ein Studie mit dem Titel How Do They Manage? An Investigation of Early Childhood
      Leadership- Wie managen sie? Eine Untersuchung von Leitungkr�ften in der fr�hkindlichen Bildung und Erziehung.
      Die Studie befasst sich mit Kita-Leitungen, die unterschiedliche Qualifikationsniveaus haben.\citep[vgl.][S.~5]{How2013}
      Die Kita-Leitungen gaben an, welche Kompetenzen ben�tigt werden um als Kita-Leitung effektiv zu fungieren.\citep[vgl.][S.~45]{LeitungvKi2014}
      Eine nicht aus Gro�britannien stammende Studie ist die von der finnischen Professorin Eeva Hujala.
      Die Professsorin befasst sich seit vielen Jahren mit dem Themen Kita-Leitungen, Qualit�t und die
      p�dagogische Arbeit in der fr�hkindlichen Bildung und Erziehung. Zu diesen Themen verfasste sie Literatur
      (Researching Leadership in Early Childhood Education, Contextually Defined Leadership, Shared Leadership
      among Caribbean Early Childhood Practitioners, Dimensions of leadership in the childcare context...) f�hrt
      und f�hrte Studien zu ihnen durch. Ergebnisse ihrer Studie, die Teil eines internationalen Projektes waren,
      sind unter anderem dass unterschiedliche Personengruppen unterschiedliche Erwartungen an Kita-Leitungen haben.\citep[vgl.][S.~67]{DimLeader2004}
      Die Eltern erwarten einen gewissen Service und eine gute Zusammenarbeit. W�hrend die Tr�ger und Politik vor
      allem erwarteten, dass die Leitung das Personal f�hrt und dem Bildungsauftrag gerecht wird. Die
      Mitarbeiterinnen selber glaubten, dass der Auftrag einer Kita-Leitung sei es ihr Team und jeden einzelnen
      zu unterst�tzen und die zwischenmenschlichen Beziehungen zu pflegen. Au�erdem sollten ihrer Meinung nach
      Leitungskr�fte vorrausschauend handeln und denken und den zusammenhalt im Team gew�hren. Auch die Kita-Leitungen
      selbst gaben im Laufe der Studie an, was sie denken was eine Leitungskraft zu bew�ltigen hat. So war f�r sie
      ebenfalls der Zusammenhalt im Team von hoher Priorit�t, sowie die  Teamentwicklung, sich �ber neue
      Forschungsergebnisse zu informieren und das Team zu bef�higen und zu animieren gute p�dagogische Arbeit zu leisten. \citep[vgl.][S.~45]{LeitungvKi2014}
      Weitere Ergebnisse der Studie waren, dass die Kita-Leitung in ihrer Rolle als Leitung viele Rollen �bernehmen
      muss. Aus den Erwartungen an eine Kita-Leitung und den verschieden Rollen die sie hat, l�sst sich schlie�en
      dass eine Kita-Leitung eine multifunktionale Person sein sollte. Deren Aufgaben nicht wirklich klar definiert sind. \citep[vgl.][S.~65ff]{DimLeader2004}
      Sie steht im Grunde zwischen den St�hlen und muss hier bestm�glich agieren. \citep[vgl.][S.~45]{LeitungvKi2014}
    \end{flushleft}

\newpage


%---------------------------------------------------------------------------------------------------
% Der dritte Teil der Arbeit
%---------------------------------------------------------------------------------------------------
\typeout{===== File: ZUSAMMENFASSUNG}
%---------------------------------------------------------------------------------------------------
% Zusammenfassung
%---------------------------------------------------------------------------------------------------
% \newpage
%%\part{Schluss}
\chapter{Fazit und Ausblick}

  \begin{flushleft}
    Die Zielsetzung dieser Arbeit war es, den Einfluss hinsichtlich des Qualifikationsnivaus von Kita-Leitung auf die fr�hkindlichen Bildung in Kindertageseinrichtungen, zu beleuchten. Zu diesem Zweck wurde erst gekl�rt, welche Anforderungen, Aufgaben und T�tigkeiten Kita-Leitungen zu erf�llen haben. Im weiteren Verlauf wurde dann darauf eingegangen, welche Qualifikationsformen Leitungskr�fte in Deutschland innewohnen.
Dem folgte die  Auseinandersetzung mit nationalen und internationalen Forschungsergebnissen zu dem Thema Leitungskr�ften in Kindertageseinrichtungen.
  \end{flushleft}

  \begin{flushleft}
    Durch die Hausarbeit konnte gezeigt werden, welchen Stellenwert die Qualifikation auf die fr�hkindliche Bildung in Kindertageseinrichtungen hat. Gerade die internationalen Forschungsergebnisse, wie etwa die EPPE Studie, zeigen deutlich, dass ein h�her qualifizierteres Personal auch f�r mehr Qualit�t in der Einrichtung steht. Dies gilt im Besonderen, wenn die Kita-Leitung eine h�here Qualifizierung besitzt.\citep[vgl.][S.~4]{EPPE2004} Hierf�r scheint haupts�chlich das Verst�ndnis von Qualit�t und wie sie diese erreicht werden kann, verantwortlich zu sein.\citep[vgl.][S.~86]{Long2012}  Zu dem wurde aber auch deutlich gemacht, dass deutschlandweit  77\% der Kita-Leitungen keinen einschl�gigen Hochschulabschluss haben. Daher scheinen Fort- und Weiterbildungen derzeit eine gute L�sung zu sein um das Qualifikationsnivaus von Kita-Leitungen zu steigern.\citep[vgl.][S.~269]{Ausblick2014} Au"serdem scheint ein einschl�giger Hochschulabschluss nicht auszureichen, um dem Anforderungs-, Aufgaben- und T�tigkeitsprofil von Kita-Leitungen voll und ganz zu entsprechen. Daher sind auch hier Fort- und Weiterbildungen von gro"ser Bedeutung. \citep[vgl.][S.~269]{Ausblick2014}
  \end{flushleft}

  \newpage
  \begin{flushleft}
    Nun zu meinem pers�nlichen Fazit. Ich hatte mir von dieser Facharbeit erhofft, zwei Sachen zu erfahren. Zum einen war es mir wichtig zu erfahren, inwieweit das Thema in der Fach�ffentlichkeit thematisiert und diskutiert wird. Zum anderen wollte ich wissen, inwiefern zu meiner eingehenden Frage, wissenschaftliche Erkenntnisse existieren.
  \end{flushleft}

  \begin{flushleft}
    Leider wurden nicht beide Ziele zu meiner vollsten Zufriedenheit erf�llt. Ich hatte zwar im Rahmen dieser Arbeit die Gelegenheit, viele Erkenntnisse �ber die Meinungen der Fach�ffentlichkeit zu erlangen, da es mir hierbei in keiner Weise an Fachliteratur gemangelt hat. Dadurch habe ich neue Erkenntnisse �ber die Wichtigkeit von Kita-Leitungen und ihren T�tigkeiten erlangt, die mir auch in meinem weiteren Berufsleben sehr hilfreich sein werden. Was mich aber �berrascht hat, war dass es noch auf nationaler Ebene wenig wissenschaftliche Ergebnisse zu diesem Thema gibt. Zwar sind die internationalen Studien sehr informativ, doch h�tte ich mir bei der zugeschriebenen \enquote{Schl�sselposition}, mehr nationale Studien erhofft.
  \end{flushleft}

  \begin{flushleft}
    Durch die Recherche f�r die vorliegende Arbeit haben sich Weiterf�hrende Fragen ergeben - Welchen Einfluss hat die Freistellung der Kita-Leitung auf die Qualit�t der fr�hkindlichen Bildung? Welche Rolle Spielt die Freistellung? und Welche Rolle spielt das Alter einer Kita-Leitung? Ich k�nnte mir gut vorstellen, diese Fragestellungen in meiner Bachelorthesis weiterzuverfolgen und diese mit der vorliegenden Arbeit zu verkn�pfen.
  \end{flushleft}
  \begin{flushleft}
  \end{flushleft}



%%%%

% \clearpage
\cleardoublepage
% \typeout{===== Section: Anhang}
% \appendix
%
% \includepdfset{pagecommand={\pagestyle{scrheadings}}}
% \includepdf[pages=-, addtotoc={1,chapter,0,Interview mit Herrn Strau�,chap:anhang_1}, scale=0.90, offset=6mm -5mm]{Interview.pdf}
%
% \includepdfset{pagecommand={\pagestyle{scrheadings}}}
% \includepdf[pages=-, addtotoc={1,chapter,0,Interviewleitfaden zur Erforschung des Berufsfeldes,chap:anhang_2}, scale=0.90, offset=6mm -5mm]{Interviewleifaden.pdf}
%
% \includepdfset{pagecommand={\pagestyle{scrheadings}}}
% \includepdf[pages=-, addtotoc={1,chapter,0,Das Shadowing,chap:anhang_3}, scale=0.90, offset=6mm -5mm]{Shadowing.pdf}
%
% \includepdfset{pagecommand={\pagestyle{scrheadings}}}
% \includepdf[pages=-, addtotoc={1,chapter,0,Dienstplanberechnung,chap:anhang_4}, scale=0.80, offset=6mm -5mm]{anhang_4.pdf}
%
% \includepdfset{pagecommand={\pagestyle{scrheadings}}}
% \includepdf[pages=-, addtotoc={1,chapter,0,Verfassung der KiTa,chap:anhang_5}, scale=0.80, offset=6mm -5mm]{anhang_5.pdf}
%
% \includepdfset{pagecommand={\pagestyle{scrheadings}}}
% \includepdf[pages=-, addtotoc={1,chapter,0,Grundriss der KiTa,chap:anhang_6}, scale=0.80, offset=6mm -5mm]{anhang_6.pdf}
%
% \includepdfset{pagecommand={\pagestyle{scrheadings}}}
% \includepdf[pages=-, addtotoc={1,chapter,0,Beobachtungsbogen,chap:anhang_7}, scale=0.80, offset=6mm -5mm]{anhang_7.pdf}
%
% \includepdfset{pagecommand={\pagestyle{scrheadings}}}
% \includepdf[pages=-, addtotoc={1,chapter,0,Einzel Soziogramm,chap:anhang_8}, scale=0.80, offset=6mm -5mm]{anhang_8.pdf}
%
% \includepdfset{pagecommand={\pagestyle{scrheadings}}}
% \includepdf[pages=-, addtotoc={1,chapter,0,Kinderfragebogen,chap:anhang_9}, scale=0.80, offset=6mm -5mm]{anhang_9.pdf}
%
% \includepdfset{pagecommand={\pagestyle{scrheadings}}}
% \includepdf[pages=-, addtotoc={1,chapter,0,Meinungsbild Mittagessen,chap:anhang_10}, scale=0.80, offset=6mm -5mm]{anhang_10.pdf}

% bibliography and other stuff
\backmatter
% \pagenumbering{arabic}
\typeout{===== Section: Literaturverzeichnis}
\cleardoublepage
\phantomsection
\addcontentsline{toc}{chapter}{Literaturverzeichnis}
\bibliographystyle{natdin}
\bibliography{literatur}

% \typeout{===== Section: Glossar}
% \renewcommand{\glossaryname}{Glossar}
% \cleardoublepage
% \phantomsection
% \addcontentsline{toc}{chapter}{Glossar}
% \clearpage
\printglossaries

%% index
% \typeout{===== Section: Index}
\cleardoublepage
% \phantomsection
% \addcontentsline{toc}{chapter}{Index}
% \printindex






% \HAWasurency

\end{document}
